\documentclass[11pt, oneside]{article}   	% use "amsart" instead of "article" for AMSLaTeX format
\usepackage{geometry}                		% See geometry.pdf to learn the layout options. There are lots.
\usepackage{cancel}
\geometry{letterpaper}                   		% ... or a4paper or a5paper or ... 
%\topmargin = -125pt
%\textheight = 1000pt
%\footskip = 100pt

\usepackage{graphicx}				% Use pdf, png, jpg, or eps§ with pdflatex; use eps in DVI mode
								% TeX will automatically convert eps --> pdf in pdflatex		
\usepackage{amssymb}
\usepackage{bm}
\newcommand{\matrva}[1]{\bm{#1}} 

\title{\vspace{-3.0cm}Lab 3 - Continue Modeling MotorLab System}
\author{Derek Black}
\date{\vspace{-5ex}}

\begin{document}
\maketitle

%% Introduction Section %%
\section{Introduction}

%% Second Section %%
\section{Dynamic Model with Spring Added}

\[T = J \ddot \theta(t) \ + b \dot \theta(t) + k_s \theta(t) \]

%% Third Section %%
\section{MotorLab Coefficients}
\subsection{Coefficients Needed to be Found}
\begin{itemize}
\item \(\zeta\)    ('zeta' or 'damping ratio')
\item \(w_n\)     ('natural frequency')
\item \(k_s\)      ('spring constant')
\end{itemize}

\subsection{How to Find \(k_s\)}
\begin{itemize}
\item Like Lab 2, we will be looking at steady state
\item Unlike Lab 2, we have a spring attached. This means we have a constant position opposed to constant velocity like Lab 2 when we apply amperage to the motor
\[T(t) = k_t i(t) =  J \ddot \theta(t) \ + b \dot \theta(t) + k_s \theta(t) \]
\[T(t) = k_t i(t) =  J \cancelto{0}{\ddot \theta(t)}  + b \cancelto{0}{\dot \theta(t)} + k_s \theta(t) \]
\[k_t i(t) = k_s \theta(t) \] \label{eu_eqn}
\[k_s = k_i \frac{i(t)}{\theta(t)} \]

\item We will find theta by commanding current to the motor and then measuring the position of the motor shaft
\item To estimate \(k_s\), the data should come out to be linear, so all you have to do is estimate the slope to find \(k_s\)
\subitem You can play around with \(k_s\) manually to estimate slope
\subitem You can export data to excel and use linear regression
\subitem You can use the 'Normal Equation'
\[k_s = (\matrva{\theta^T \theta})^{-1} \matrva{\theta i}    \]
\end{itemize}

\subsection{How to Find \(\zeta\) - Logarithmic Decrement}
\begin{itemize}
\item Log. Decrement is used for finding the damping ratio of an underdamped system
\item Defined as the natural log of the ratio of amplitude of two peaks, n periods apart, namely \(ln(P_i/P_(i+n))\)
\item Will us this ratio to estimate the damping ratio
\item You will need to find three points using data cursors (How to do this will be covered)
\end{itemize}

\subsection{Finding \(w_n\), \(T_{osc}\), \(w_d\), b}
\begin{itemize}
\item 


\end{document}  
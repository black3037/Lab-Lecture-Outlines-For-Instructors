\documentclass[11pt, oneside]{article}   	% use "amsart" instead of "article" for AMSLaTeX format
\usepackage{geometry}                		% See geometry.pdf to learn the layout options. There are lots.
\usepackage{cancel}
\usepackage{marginnote}
\usepackage{mathtools}
\geometry{letterpaper}                   		% ... or a4paper or a5paper or ... 
\usepackage{graphicx}				% Use pdf, png, jpg, or eps§ with pdflatex; use eps in DVI mode
								% TeX will automatically convert eps --> pdf in pdflatex		
\usepackage{amssymb}
\usepackage{bm}                                       % The 2 Following Liens Allows Matrix Notation
\newcommand{\matrva}[1]{\bm{#1}} 

\title{\vspace{-3.5cm}Lab 3 - Continue Modeling MotorLab System}
\author{Derek Black}
\date{\vspace{-5ex}}

\begin{document}
\maketitle

%% Introduction Section %%
\section{Introduction}

%% Second Section %%
\section{Dynamic Model with Spring Added}

\[T = J \ddot \theta(t) \ + b \dot \theta(t) + k_s \theta(t) \]

%% Third Section %%
\section{MotorLab Coefficients}

\subsection{Coefficients Needed to be Found}
\begin{itemize}
\item \(\zeta\)    ('zeta' or 'damping ratio')
\item \(w_n\)     ('natural frequency')
\item \(k_s\)      ('spring constant')
\end{itemize}

\subsection{How to Find \(k_s\)}
\begin{itemize}
\item Like Lab 2, we will be looking at steady state
\item Unlike Lab 2, we have a spring attached. This means we have a constant position opposed to constant velocity like Lab 2 when we apply amperage to the motor
\[T(t) = k_t i(t) =  J \ddot \theta(t) \ + b \dot \theta(t) + k_s \theta(t) \]
\[T(t) = k_t i(t) =  J \cancelto{0}{\ddot \theta(t)}  + b \cancelto{0}{\dot \theta(t)} + k_s \theta(t) \]
\[k_t i(t) = k_s \theta(t) \] \label{eu_eqn}
\[k_s = k_t \frac{i(t)}{\theta(t)}  = \frac{T(t)}{\theta(t)}\] 
\item This means we can find \(k_s\) from the slope of \(T(t)\) and \(\theta(t)\)

\item We will find theta by commanding current to the motor and then measuring the position of the motor shaft (Demonstrated in Lab)
\item To estimate \(k_s\), the data should come out to be linear, so all you have to do is estimate the slope to find \(k_s\)
\subitem You can play around with \(k_s\) manually to estimate slope
\subitem You can export data to excel and use linear regression
\subitem You can use the 'Normal Equation'
\[\footnote{ \(\matrva{\theta}\) and \(\matrva{T}\) defined already in code after you have ran the experiment and put in values for position}k_s = (\matrva{\theta \theta^T})^{-1} \matrva{\theta T^T}    \]
\end{itemize}

%\marginnote{You have \(\matrva{\theta}\) and \(\matrva{T}\) defined already in MATLAB}[-0.1cm]

\subsection{How to Find \(\zeta\) - Logarithmic Decrement}
\begin{itemize}
\item Log. Decrement is used for finding the damping ratio of an underdamped system
\item Defined as the natural log of the ratio of amplitude of two peaks, n periods apart, namely \(ln(P_i/P_{i+n})\)
\item Will us this ratio to estimate the damping ratio
\item You will need to find three points using data cursors in MATLAB (How to do this will be covered)
\end{itemize}

\subsection{Finding \(w_n\), \(T_{osc}\), \(w_d\), b}
\begin{itemize}
\item We can find rest of the coefficients of the MotorLab by equating a standard \(2^{nd}\) order system with our model

\[ G(s) = 
\underbrace{\frac{k_t}{Js^2 + bs + k_s}}_\text{Motorlab Model} = 
\underbrace{\frac{k_{dc}w_n^2 }{s^2 + 2\zeta w_n s + w_n^2}}_\text{Standard \(2^{nd}\) Order System}  \]

\item Equate the coefficients of the two equations to solve for \(w_n\), b, etc.
\item REMEMBER: THE COEFFICIENTS ARE NOT EQUIVILENT UNTIL YOU TAKE CARE OF J IN FRONT OF \(s^2\) IN THE MOTORLAB MODEL! 

\end{itemize}

\section{Final Thoughts}
\begin{itemize}
\item Make sure you've read the entire lab. It has information that will help you finish this lab and fill out the code
\item We are looking for numeric answers for the coefficients table, not equations
\item Have the instructors sign off the questions before you leave
\end{itemize}

\end{document}  